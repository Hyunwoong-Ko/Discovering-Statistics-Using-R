\documentclass{article}[14pt]
\usepackage{amsmath}
\usepackage{amsfonts}
\usepackage{graphicx}
\usepackage{enumerate}
\usepackage{dtklogos}
\usepackage{verbatim}
\usepackage{url}
\usepackage{natbib}
\usepackage{calrsfs}
\usepackage{collectbox}
\newcommand{\R}{{\mathbb R}}
\renewcommand{\vec}[1]{{\mathbf #1}}
\newcommand{\points}[1]{\phantom{.}\hfill \textbf{(#1 points)}}
\newcommand{\matlab}{{\textsc{Matlab}} }


\begin{document}

\hfill Iliass Tiendrebeogo\\

\hfill \date{Jannuary 25, 2016} \\

\bigskip

\begin{center}
  \begin{Large}
    Math 567: Homework 2\\
    Jannuary 25, 2016\\
   
  \end{Large}
\end{center} 


\begin{enumerate}[1.]
\item  % 1 
{\bf The degree to which a statistical model represents the data collected is known as the: }
 \begin{enumerate}[A.]
 \item 
  Fit
\end{enumerate}
\item %2
{\bf ‘Children can learn a second language faster before the age of 7’. Is this statement: }
 \begin{enumerate}[D.]
 \item 
  A one-tailed hypothesis.
\end{enumerate}

\item %3
 {\bf If a psychological test is valid, what does this mean?}
 \begin{enumerate}[B.]
 \item 
  The test measures what it claims to measure
  \end{enumerate}

\item %4
 {\bf If my null hypothesis is ‘Dutch people do not di↵er from English people in height’, what is my alternative
hypothesis?}
 \begin{enumerate}[C.]
 \item 

  Dutch people differ in height from English people
  \end{enumerate}
 \item %5
 {\bf  When questionnaire scores predict, or correspond with, external measures of the same construct that
the questionnaire measures it is said to have:}
 \begin{enumerate}[C.]
 \item 
  Content validity
\end{enumerate}

 \item %6
  {\bf A variable manipulated by a researcher is known as:}
 \begin{enumerate}[C.]
 \item 
  A discrete variable.
\end{enumerate}
 \item %7
  {\bf  What kind of variable is IQ, measured by a standard IQ test?}
 \begin{enumerate}[A.]
 \item 
  Categorical.
\end{enumerate}
 \item %8
  {\bf  A frequency distribution in which high scores are most frequent (i.e. bars on the graph are highest on the right-hand side) is said to be:}
 \begin{enumerate}[D.]
 \item 
   Negatively skewed.
\end{enumerate}

 \item %9
   {\bf A frequency distribution in which there are too few scores at the extremes of the distribution is said to be:  }
 \begin{enumerate}[B.]
 \item 
   Leptokurtic
\end{enumerate}
 \item %10
    {\bf  Which of the following is designed to compensate for practice effects?}
 \begin{enumerate}[B.]
 \item 
   Randomization of participants
\end{enumerate}

\end{enumerate}
\end{document}