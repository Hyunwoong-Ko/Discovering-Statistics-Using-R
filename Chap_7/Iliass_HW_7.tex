\documentclass{article}[20pt]
\usepackage{amsmath}
\usepackage{amsfonts}
\usepackage{graphicx}
\usepackage{enumerate}
\usepackage{dtklogos}
\usepackage{verbatim}
\usepackage{url}
\usepackage{natbib}
\usepackage{calrsfs}
\usepackage{collectbox}
\newcommand{\R}{{\mathbb R}}
\renewcommand{\vec}[1]{{\mathbf #1}}
\newcommand{\points}[1]{\phantom{.}\hfill \textbf{(#1 points)}}
\newcommand{\matlab}{{\textsc{Matlab}} }


\begin{document}

\hfill Iliass Tiendrebeogo\\

\hfill \today\\

\bigskip

\begin{center}
  \begin{Large}
    Math 567: Homework 7 \\
    
   
  \end{Large}
\end{center} 


\begin{enumerate}[1.]
\item  % 1 
{\bf$ R^2 $}
 \begin{enumerate}[D.]
 \item 
The proportion of variance in the outcome accounted for by the predictor variable or variables..
\end{enumerate}
\item %2
{\bf .}
 \begin{enumerate}[D.]
 \item 
  The t-statistic is equal to the regression coefficient divided by its standard deviation.
  \end{enumerate}

\item %3
{\bf .}
 \begin{enumerate}[D.]
 \item 
  Outliers are influential cases.
  \end{enumerate}

\item %4
 {\bf The Durbin–Watson statistic test}
 \begin{enumerate}[B.]
 \item 
  \texttt{ Independence of errors.}
  
 \end{enumerate}
 
 \item %5
{\bf Figure 1 Shows:}
 \begin{enumerate}[C.]
 \item 
  \texttt{ Linearity }
  
 \end{enumerate}

\end{enumerate}
\end{document}