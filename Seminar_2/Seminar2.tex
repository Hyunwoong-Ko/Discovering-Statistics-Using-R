\documentclass{article}[12pt]
\usepackage{amsmath}
\usepackage{amsfonts}
\usepackage{graphicx}
\usepackage{enumerate}
\usepackage{dtklogos}
\usepackage{verbatim}
\usepackage{url}
\usepackage{natbib}
\usepackage{calrsfs}
\usepackage{collectbox}
\usepackage{blindtext}
\newcommand{\R}{{\mathbb R}}
\renewcommand{\vec}[1]{{\mathbf #1}}
\newcommand{\points}[1]{\phantom{.}\hfill \textbf{(#1 points)}}
\newcommand{\matlab}{{\textsc{Matlab}} }


\begin{document}
\begin{center}


\title{Exploring Assumption - Seminar 2}
\hfill Iliass Tiendrebeogo\\

\hfill \today\\
\end{center}
\bigskip

\begin{center}
  \begin{Large}
      
    Exploring Assumption - Seminar 2 \\
    Math 567: Winter 2016 \\
       
  \end{Large}
\end{center}

\bigskip
\begin{enumerate}
\item % 1
{\bf Shapiro-Wilk Test}

$$ W = \frac{(\sum_{i=1}^{n} a_i x_{(i)} )^2}{\sum_{i=1}^{n} (x_i - )^2 } $$


\item % 2
{ \bf Levene's Test} Variance Homogeneity Test

In Statistical evaluation Leven's Test is an Analysis of Variance (ANOVA) test that measures the equality of variances of variables between two or more groups. It is the absolute difference between each score and the mean of the group from which it came. Here the null hypothesis assume the homogeneity of the variances. If Leven's Test is significant, p-value < .05, we can reject the null hypothesis, otherwise we accept the assumption of homogeneity.

\item % 1
{\bf Hartley's Test}
\end{enumerate}


\end{document}