\documentclass{article}[16pt]
\usepackage{amsmath}
\usepackage{amsfonts}
\usepackage{graphicx}
\usepackage{enumerate}
\usepackage{dtklogos}
\usepackage{verbatim}
\usepackage{url}
\usepackage{natbib}
\usepackage{calrsfs}
\usepackage{collectbox}
\newcommand{\R}{{\mathbb R}}
\renewcommand{\vec}[1]{{\mathbf #1}}
\newcommand{\points}[1]{\phantom{.}\hfill \textbf{(#1 points)}}
\newcommand{\matlab}{{\textsc{Matlab}} }


\begin{document}

\hfill Iliass Tiendrebeogo\\

\hfill ID:30742742\\

\bigskip

\begin{center}
  \begin{Large}
    Math 567: Homework 6 \\
     {\small \today}
   
  \end{Large}
\end{center} 


\begin{enumerate}[1.]
\item  % 4
{\bf Covariance}
 \begin{enumerate}[B.]
 \item
A measure of the strength of relationship between two variables.
\end{enumerate}
\item %5
{\bf Children can learn a second language faster before the age of 7. Is this statement:}
 \begin{enumerate}[A.]
 \item 
  It can be used as an effect size measure.
  \end{enumerate}

\item %6
{\bf How much variance has been explained by a correlation of .9?}
 \begin{enumerate}[A.]
 \item 
81\%
  \end{enumerate}

\item %7
{\bf The relationship between two variables controlling for the effect that a third variable has on one of those
variables can be expressed using a:}
 \begin{enumerate}[D.]
 \item 
Partial Correlation.
  \end{enumerate}
  
  

\end{enumerate}
\end{document}