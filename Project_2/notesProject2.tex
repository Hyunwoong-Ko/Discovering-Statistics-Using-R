\documentclass{article}[14pt]
\usepackage{amsmath}
\usepackage{amsfonts}
\usepackage{graphicx}
\usepackage{enumerate}
\usepackage{dtklogos}
\usepackage{verbatim}
\usepackage{url}
\usepackage{natbib}
\usepackage{calrsfs}
\usepackage{collectbox}
\newcommand{\R}{{\mathbb R}}
\renewcommand{\vec}[1]{{\mathbf #1}}
\newcommand{\points}[1]{\phantom{.}\hfill \textbf{(#1 points)}}
\newcommand{\matlab}{{\textsc{Matlab}} }


\begin{document}



\bigskip

\begin{center}
  \begin{Large}
    Project 2\\
    2013 period
  \end{Large}
\end{center} 

\section{Plots Jan.-June 2013}

\subsection{Apple Chart}

The plot of Apple stock in Figure 3.29 show that Apple stock price decreased quickly from January. to April 2013 where it hit a minimum of \$425 losing \$100. It then stabilize with a little gain until mid May, then re plunge for the rest of the second quarter to hit a low of \$418.
In overall the Apple stock price lost \$107 during the first 6 months of 2013.


\subsection{Microsoft Chart}

In first 2 quarters of 2013, Microsoft stock price plot has a "s" shape. This stock increased with a low slop from January at \$27 to April 2013, then suddenly quickly increase with a faster slop until June and stabilize from there at \$34.9. 
\subsection{Correlation}
When we evaluate the correlation between Apple , Microsoft, Nasdaq index and Nasdaq-Composite, we found that Apple and Microsoft stock are inversely correlated. While Microsoft and Nasdaq and Nasdaq-Composite are strongly correlated with a correlation coefficient of 0.95. Apple is inversely correlated to Nasdaq and Nasdaq-Composite.
\subsection{Regression}
Adding a regression line on the Apple stock chart it is obvious that the stock decreased on overall during this period. Microsoft stock chart, in the opposite side has a steep positive slope, hence its fast increase.
\subsection{Analysis}
The analysis of Microsoft an Apple stock show that while Apple stock was going down, Microsoft stock was going up at inversely pace. In fact the opposite correlation of Apple stock with Nasdaq index confirmed the fact that Apple stock was going on the opposite direction of the market.

\section{Plots July- December 2013}

\subsection{Apple Chart}

The plot of Apple stock show from July to September a rapid increase from \$400 to \$490, then it flatten at that price for a month, and continue increasing faster to reach \$560 at the end of the 2013 year. Apple stock gain \$160 in 6 month period.

\subsection{Microsoft Chart}

Microsoft stock see a decreased from \$35 July to mid August by hitting a low \$32. From mid August to end November this stock re-bounced an attained an high of \$37.5 with \$5 gain.
\subsection{Correlation}
When we evaluate the correlation between Apple , Microsoft, Nasdaq index and Nasdaq-Composite, we found a fair correlation between  Apple and Microsoft stocks. While Microsoft Nasdaq and Nasdaq-Composite are correlated with a correlation coefficient of 0.77, Apple  Nasdaq and Nasdaq-Composite are strongly correlated at 0.9. In overall all these stock have fairly strong correlation.
\subsection{Regression}
The regression line added on the Apple stock chart show positive slope of about 45 degree which confirms its gain through the last 6 months.

Microsoft stock chart regression line also has a positif slope of roughly 45 degree. At last both regression models are capturing the overall progression of the stock at this period. 

\subsection{Analysis}
The analysis of Microsoft an Apple stock during the last 6 months of 2013 show that in overall both stock have had value even though Apple went up the most. The strong correlation with Nasdaq index is confirmation that none of these stock went against the market. At the end both stock byers was happy. 

\end{document}